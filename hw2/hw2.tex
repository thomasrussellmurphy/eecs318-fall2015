\documentclass[10pt]{article}

\usepackage[psamsfonts]{amsfonts}
\usepackage{amsmath}
\usepackage{amssymb,latexsym}
\usepackage{amsthm}
\usepackage{bm}
\usepackage{graphicx}
\usepackage[T1]{fontenc}

\usepackage[left=2.35cm,top=2.45cm,bottom=2.45cm,right=2.35cm,letterpaper]{geometry}

\usepackage{hyperref}
\usepackage{color}

\usepackage{fancyhdr}
\pagestyle{fancy}

\fancyhf{}
\fancyfoot[C,CO]{\thepage}
\fancyhead[L,LO]{\bfseries EECS 318}
\fancyhead[C,CO]{\bfseries Homework 2}
\fancyhead[R,RO]{Fall 2015}

\theoremstyle{definition}
\newtheorem{problem}{Problem}
\newtheorem*{solution}{Solution}

\newcommand{\N}{\mathbb{N}}
\newcommand{\Z}{\mathbb{Z}}
\newcommand{\Q}{\mathbb{Q}}
\newcommand{\R}{\mathbb{R}}
\newcommand{\C}{\mathbb{C}}

\renewcommand{\labelenumi}{\textbf{(\alph{enumi})}}
\renewcommand{\labelenumii}{\textbf{(\roman{enumii})}}

\newcommand{\ds}{\displaystyle}

\newcommand{\ttname}[1]{\texttt{\detokenize{#1}}}

\parindent=0pt

\begin{document}

  \begin{center}
  \section*{Homework 2}
  \end{center}
  Thomas Murphy (\textit{trm70})

  \today

  \bigskip

  \begin{problem}
    Synchronous Serial Port

    The SSP is implemented in the following modules:

    \begin{itemize}
      \item Transmit FIFO: \ttname{q1/tx_fifo.v}
      \item Receive FIFO: \ttname{q1/rx_fifo.v}
      \item Transmit and Receive Logic: \ttname{q1/ssp_tx_rx.v}
      \item SSP Top-Level: \ttname{q1/ssp.v}
    \end{itemize}

    The testbenches to support this design are:

    \begin{itemize}
      \item For transmit FIFO: \ttname{q1/tb_tx_fifo.v}
      \item For receive FIFO: \ttname{q1/tb_rx_fifo.v}
      \item For top-level SSP: \ttname{q1/ssptest.v} (containing two modules provided by instructor)
    \end{itemize}
  \end{problem}

  \begin{problem}
    Processor Implementation and Negation Program Test

    The processor described in the problem is implemented in the single module:

    \begin{itemize}
      \item \ttname{q2/processor.v}
    \end{itemize}

    This covers all of the implementation components required by the test programs.

    The negation program implementation is in:

    \begin{itemize}
      \item Testbench loading: \ttname{q2/tb_processor_q2.v}
      \item Program and data memory: \ttname{memory1.list}, a plain text file in Verilog binary memory format
    \end{itemize}
  \end{problem}

  \begin{problem}
    Counting 1s Program Test

    The 1s count program implementation is in:

    \begin{itemize}
      \item Testbench loading: \ttname{q3/tb_processor_q3.v}
      \item Program and data memory: \ttname{memory2.list}, a plain text file in Verilog binary memory format
    \end{itemize}

    The testbench depends on the original processor, \ttname{q2/processor.v}.
  \end{problem}

  \begin{problem}
    Multiplication Program Test

    The multiplication program implementation is in:

    \begin{itemize}
      \item Testbench loading: \ttname{q4/tb_processor_q4.v}
      \item Program and data memory: \ttname{memory3.list}, a plain text file in Verilog binary memory format
    \end{itemize}
  \end{problem}

\end{document}
