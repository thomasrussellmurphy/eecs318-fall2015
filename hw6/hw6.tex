\documentclass[10pt]{article}

\usepackage[psamsfonts]{amsfonts}
\usepackage{amsmath}
\usepackage{amssymb,latexsym}
\usepackage{amsthm}
\usepackage{bm}
\usepackage{graphicx}
\usepackage[T1]{fontenc}

\usepackage[left=2.35cm,top=2.45cm,bottom=2.45cm,right=2.35cm,letterpaper]{geometry}

\usepackage{hyperref}
\usepackage{color}

\usepackage{fancyhdr}
\pagestyle{fancy}

\fancyhf{}
\fancyfoot[C,CO]{\thepage}
\fancyhead[L,LO]{\bfseries EECS 318}
\fancyhead[C,CO]{\bfseries Homework 6}
\fancyhead[R,RO]{Fall 2015}

\theoremstyle{definition}
\newtheorem{problem}{Problem}
\newtheorem*{solution}{Solution}

\newcommand{\N}{\mathbb{N}}
\newcommand{\Z}{\mathbb{Z}}
\newcommand{\Q}{\mathbb{Q}}
\newcommand{\R}{\mathbb{R}}
\newcommand{\C}{\mathbb{C}}

\renewcommand{\labelenumi}{\textbf{(\alph{enumi})}}
\renewcommand{\labelenumii}{\textbf{(\roman{enumii})}}

\newcommand{\ds}{\displaystyle}

\newcommand{\ttname}[1]{\texttt{\detokenize{#1}}}

\parindent=0pt

\begin{document}

  \begin{center}
  \section*{Homework 6}
  \end{center}
  Thomas Murphy (\textit{trm70})

  \today

  \bigskip

  \begin{problem}
    ISCAS Benchmark Simulator and Performance Tester

    The input format processor and simulator is implemented in \ttname{src/main.cpp}. The base of the processor was taken from HW5. In should be compiled using the provided \ttname{Makefile} to ensure the correct compiler configuration is used. This program makes heavy use of \ttname{C++11} features. The two provided input sets are located in \ttname{inputs/}.

    \bigskip

    The simulator is run using the command \ttname{main netlist.txt inputs.vec outfile [s/t]}. The two input filenames are the ISCAS benchmark file and a plain-text table of the value for each primary input declared in the ISCAS format. The simulation state after each input is processed is written to a file with the given name. The simulator uses either \textbf{t}able-lookup or input-\textbf{s}canning depending on the flag given.

    \bigskip

    The output file for the \ttname{inputs/s27.txt} input circuit with the \ttname{inputs/s27.vec} input vectors is as follows. In table-lookup mode, the simulation takes 0.000819224s. In input-scanning mode, the simulation takes 0.000650427s.

    \begin{verbatim}
Input  :0000
State  :XXX
Output :X

Input  :0010
State  :0XX
Output :X

Input  :0100
State  :0X0
Output :X

Input  :1000
State  :0X1
Output :1

Input  :1111
State  :101
Output :1


    \end{verbatim}

    \bigskip

    The ~500kB output file for the \ttname{inputs/s35932.txt} input circuit with the \ttname{inputs/s35932.vec} is not included in this document. In table-lookup mode, the simulation takes 366.31s. In input-scanning mode, the simulation takes 214.304s.

    \bigskip

    For this implementation of the two simulation techniques, it appears that the input-scanning technique is faster than the table-lookup technique.

  \end{problem}

\end{document}
